% 04.tex
\documentclass[platex,dvipdfmx]{jsreport}

\usepackage{graphicx}
\graphicspath{{./images/}{../images/}}
\usepackage{pdfpages}
\usepackage{tikz}
\usepackage{xcolor}
\definecolor{UD_GREEN}{HTML}{03af7a}
\usepackage{bm}
\usepackage[left=30truemm]{geometry}
\usepackage{amsmath,amssymb}
\numberwithin{equation}{section}


\begin{document}

\chapter{結論}

誘電率$\epsilon$を大きくすればするほどギャップ-ミッドギャップは大きくなることがわかった。しかし、バンドギャップの生じる範囲については差は見られなかった。

ギャップ-ミッドギャップ比が最大となるのは$\epsilon = 10$のとき$r / a = 0.3554$、$\epsilon = 13$のとき、$r / a = 0.3580$、$\epsilon = 15$のときは$r / a = 0.3601$だった。これはいずれも逆オパール構造において最近接球同士が接するときの半径$r / a = \sqrt{2} / 4 \simeq 0.3535$よりも大きい値である。これは、バンドギャップの形成において、空気球同士をつなぐ気孔が作用しているからであると考えられる。

また、ギャップマップにおいては半径$r / a$が大きくなるほど周波数$\omega$が大きくなるような形状だった。これは、誘電率$\epsilon$の誘電体媒質中では周波数は$1 / \sqrt{\epsilon} $倍されるため、空気球の領域が増えるに従い、周波数が大きくなっていると考えられる。

% \section{ウッドパイル構造}
% ウッドパイル構造では、誘電体棒幅$w / a = 0.28, 0.29$で最大となった。

% \section{ヤブロノバイト構造}


% \section{2次元結晶の積み重ねにより作成された3次元構造}



\end{document}