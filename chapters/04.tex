%% 04.tex
\documentclass[platex,dvipdfmx]{jsreport}

\usepackage{graphicx}
\graphicspath{{./images/}{../images/}}
\usepackage{pdfpages}
\usepackage{tikz}
\usepackage{xcolor}
\definecolor{UD_GREEN}{HTML}{03af7a}
\usepackage{bm}
\usepackage[left=30truemm]{geometry}
\usepackage{amsmath,amssymb}
\numberwithin{equation}{section}


\begin{document}

\chapter{結論}
すべての構造について言えることだが、結晶内の誘電体の誘電率が異なっていてもギャップの生じる範囲はあまり変わらずギャップマップの形状は非常に似通っていた。これを利用することで誘電率を細かく変化させたときにどこにギャップが生じるのか予想できる。



\end{document}