% 04.tex
\documentclass[platex,dvipdfmx]{jsreport}

\usepackage{graphicx}
\graphicspath{{./images/}{../images/}}
\usepackage{pdfpages}
\usepackage{tikz}
\usepackage{xcolor}
\definecolor{UD_GREEN}{HTML}{03af7a}
\usepackage{bm}
\usepackage[left=30truemm]{geometry}
\usepackage{amsmath,amssymb}
\numberwithin{equation}{section}


\begin{document}

\chapter{結論}

\section{逆オパール構造}
誘電率$\epsilon$を大きくすればするほどギャップ-ミッドギャップは大きくなることがわかった。しかし、バンドギャップの生じる範囲については差は見られなかった。

逆オパール構造では球の半径$r / a$が$0.35, 0.36$の間で最大となった。これは、逆オパール構造において最近接球同士が接するときの半径$r / a = \sqrt{2} / 4 \simeq 0.3535$付近である。これは、バンドギャップの形成において、空気球同士をつなぐ気孔が作用しているからであると考えられる。



\section{ウッドパイル構造}
ウッドパイル構造では、誘電体棒幅$w / a = 0.28, 0.29$で最大となった。

\section{ヤブロノバイト構造}


\section{2次元結晶の積み重ねにより作成された3次元構造}



\end{document}